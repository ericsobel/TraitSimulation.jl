\documentclass[12pt]{article}
\usepackage{etoolbox}
\usepackage{bbm}
\usepackage{amsmath,amsfonts,amssymb}
\usepackage{mathabx}
\usepackage{graphicx}
\usepackage{float}
\usepackage{parskip}
\usepackage{indentfirst}
\usepackage{hyperref}
\usepackage{caption}
\usepackage{subcaption}
\usepackage{setspace}
\usepackage[margin=1in]{geometry}
\usepackage{booktabs}
\usepackage{rotating}
\usepackage{authblk}
\usepackage{color}
\setlength{\parskip}{0em}
\setlength{\parindent}{2em}
\floatplacement{figure}{H}
\usepackage[normalem]{ulem}
\usepackage[super,comma,sort]{natbib}
\usepackage{verbatim}

\def\E{\mathop{\rm E\,\!}\nolimits}
\def\Var{\mathop{\rm Var}\nolimits}
\def\Cov{\mathop{\rm Cov}\nolimits}
\def\Cor{\mathop{\rm Cor}\nolimits}
\def\den{\mathop{\rm den}\nolimits}
\def\midd{\mathop{\,|\,}\nolimits}
\def\sgn{\mathop{\rm sgn}\nolimits}
\def\sinc{\mathop{\rm sinc}\nolimits}
\def\curl{\mathop{\rm curl}\nolimits}
\def\div{\mathop{\rm div}\nolimits}
\def\tr{\mathop{\rm tr}\nolimits}
\def\len{\mathop{\rm len}\nolimits}
\def\vec{\mathop{\rm vec}\nolimits}
\def\diag{\mathop{\rm diag}\nolimits}
\def\dist{\mathop{\rm dist}\nolimits}
\def\Pen{\mathop{\rm Pen}\nolimits}
\def\Prior{\mathop{\rm Prior}\nolimits}
\def\Tran{\mathop{\rm Tran}\nolimits}
\def\prox{\mathop{\rm prox}\nolimits}
\def\rank{\mathop{\rm rank}\nolimits}
\def\argmin{\mathop{\rm argmin}\nolimits}
\newtheorem{proposition}{Proposition}[section]
\newtheorem{example}{Example}[section]
\newcommand{\svskip}{\vspace{1.75mm}}
\newcommand{\amp}{\mathop{\:\:\,}\nolimits}
\newcommand{\ba}{\boldsymbol{a}}
\newcommand{\bb}{\boldsymbol{b}}
\newcommand{\bc}{\boldsymbol{c}}
\newcommand{\bd}{\boldsymbol{d}}
\newcommand{\be}{\boldsymbol{e}}
\newcommand{\bff}{\boldsymbol{f}}
\newcommand{\bg}{\boldsymbol{g}}
\newcommand{\bh}{\boldsymbol{h}}
\newcommand{\bi}{\boldsymbol{i}}
\newcommand{\bj}{\boldsymbol{j}}
\newcommand{\bk}{\boldsymbol{k}}
\newcommand{\bl}{\boldsymbol{l}}
\newcommand{\bmm}{\boldsymbol{m}}
\newcommand{\bn}{\boldsymbol{n}}
\newcommand{\bo}{\boldsymbol{o}}
\newcommand{\bp}{\boldsymbol{p}}
\newcommand{\bq}{\boldsymbol{q}}
\newcommand{\br}{\boldsymbol{r}}
\newcommand{\bs}{\boldsymbol{s}}
\newcommand{\bt}{\boldsymbol{t}}
\newcommand{\bu}{\boldsymbol{u}}
\newcommand{\bv}{\boldsymbol{v}}
\newcommand{\bw}{\boldsymbol{w}}
\newcommand{\bx}{\boldsymbol{x}}
\newcommand{\by}{\boldsymbol{y}}
\newcommand{\bz}{\boldsymbol{z}}
\newcommand{\bA}{\boldsymbol{A}}
\newcommand{\bB}{\boldsymbol{B}}
\newcommand{\bC}{\boldsymbol{C}}
\newcommand{\bD}{\boldsymbol{D}}
\newcommand{\bE}{\boldsymbol{E}}
\newcommand{\bF}{\boldsymbol{F}}
\newcommand{\bG}{\boldsymbol{G}}
\newcommand{\bH}{\boldsymbol{H}}
\newcommand{\bI}{\boldsymbol{I}}
\newcommand{\bJ}{\boldsymbol{J}}
\newcommand{\bK}{\boldsymbol{K}}
\newcommand{\bL}{\boldsymbol{L}}
\newcommand{\bM}{\boldsymbol{M}}
\newcommand{\bN}{\boldsymbol{N}}
\newcommand{\bO}{\boldsymbol{O}}
\newcommand{\bP}{\boldsymbol{P}}
\newcommand{\bQ}{\boldsymbol{Q}}
\newcommand{\bR}{\boldsymbol{R}}
\newcommand{\bS}{\boldsymbol{S}}
\newcommand{\bT}{\boldsymbol{T}}
\newcommand{\bU}{\boldsymbol{U}}
\newcommand{\bV}{\boldsymbol{V}}
\newcommand{\bW}{\boldsymbol{W}}
\newcommand{\bX}{\boldsymbol{X}}
\newcommand{\bY}{\boldsymbol{Y}}
\newcommand{\bZ}{\boldsymbol{Z}}
\newcommand{\T}{\intercal}
\newcommand{\bzero}{\boldsymbol{0}}
\newcommand{\balpha}{\boldsymbol{\alpha}}
\newcommand{\bbeta}{\boldsymbol{\beta}}
\newcommand{\bgamma}{\boldsymbol{\gamma}}
\newcommand{\bdelta}{\boldsymbol{\delta}}
\newcommand{\bepsilon}{\boldsymbol{\epsilon}}
\newcommand{\blambda}{\boldsymbol{\lambda}}
\newcommand{\bmu}{\boldsymbol{\mu}}
\newcommand{\bnu}{\boldsymbol{\nu}}
\newcommand{\bphi}{\boldsymbol{\phi}}
\newcommand{\bpsi}{\boldsymbol{\psi}}
\newcommand{\bpi}{\boldsymbol{\pi}}
\newcommand{\bsigma}{\boldsymbol{\sigma}}
\newcommand{\btheta}{\boldsymbol{\theta}}
\newcommand{\bomega}{\boldsymbol{\omega}}
\newcommand{\bxi}{\boldsymbol{\xi}}
\newcommand{\bGamma}{\boldsymbol{\Gamma}}
\newcommand{\bDelta}{\boldsymbol{\Delta}}
\newcommand{\bTheta}{\boldsymbol{\Theta}}
\newcommand{\bLambda}{\boldsymbol{\Lambda}}
\newcommand{\bXi}{\boldsymbol{\Xi}}
\newcommand{\bPi}{\boldsymbol{\Pi}}
\newcommand{\bSigma}{\boldsymbol{\Sigma}}
\newcommand{\bUpsilon}{\boldsymbol{\Upsilon}}
\newcommand{\bPhi}{\boldsymbol{\Phi}}
\newcommand{\bPsi}{\boldsymbol{\Psi}}
\newcommand{\bOmega}{\boldsymbol{\Omega}}
\newcommand{\real}{\mathbb{R}}

\usepackage{listings}
\usepackage[usenames,dvipsnames]{xcolor}

%%
%% Julia definition (c) 2014 Jubobs
%%
\lstdefinelanguage{Julia}%
  {morekeywords={abstract,break,case,catch,const,continue,do,else,elseif,%
      end,export,false,for,function,immutable,import,importall,if,in,%
      macro,module,otherwise,quote,return,switch,true,try,type,typealias,%
      using,while},%
   sensitive=true,%
   alsoother={$},%
   morecomment=[l]\#,%
   morecomment=[n]{\#=}{=\#},%
   morestring=[s]{"}{"},%
   morestring=[m]{'}{'},%
}[keywords,comments,strings]%

\lstset{%
    language         = Julia,
    basicstyle       = \ttfamily,
    keywordstyle     = \bfseries\color{blue},
    stringstyle      = \color{magenta},
    commentstyle     = \color{ForestGreen},
    showstringspaces = false,
}


%-----------------------------------------------------------------------------


\begin{document}

\title{Trait simulation}
\author{Huwenbo Shi}
\maketitle

\begin{spacing}{1.0}

\section{Model (GLM / GLMM)}

\begin{equation*}
    \bY \sim \text{Dist} \left( \text{Link}^{-1} \left( \bmu = \bX{\bbeta} +
    \sum_{i=1}^k \bZ_{i} \bu_{i} \right), \text{\;other distribution-specific parameters} \right)
\end{equation*}

\begin{itemize}
    \item $\bX\bbeta$: Fixed effect component
    \item $\sum_{i=1}^k \bZ_{i} \bu_{i}$: $k$ random effect component, $k$ can be 0
    \item If $k$ is non-zero, then draw $\bmu$ from $N(\bX\bbeta, \sum_{i=1}^k \bC_i \otimes \bA_{i})$,
          where $\bC_i$ are the cross covariance matrices, and $\bA_{i}$ are the covariance matrices.
    \item Other distribution-specific parameters can be $\sigma^2$ for Gaussian distribution,
          $N$ for binomial distribution, etc.
\end{itemize}

\section{Implementation}

\begin{lstlisting}
"""
A type to store simulation parameters.
"""
type Model
  """
  Specify which distribution of the response:
    1) Binomial 2) Gamma 3) Normal 4) Poisson 5) Exponential
    6) Inverse Gaussian 7) Bernoulli etc.
  """
  distribution::AbstractString

  """
  Additional parameters for the distribution, e.g. variance for normal,
  N for binomial, etc.
  """
  parameters::Vector{Float64}

  """
  Specify the link function, GLM.jl currently supports:
  1) CauchitLink 2) CloglogLink 3) IdentityLink 4) InverseLink
  5) LogitLink 6) LogLink 7) ProbitLink 8) SqrtLink
  """
  link::AbstractString

  """
  Specify the formula of the simulation, e.g. TC ~ AGE + SNP1*SNP2 + HDL
  Using Formula of DataFrame.jl?
  """
  formula::Formula

  """
  Coefficient for each term in the formula
  """
  coefficients::Vector{Float64}
  
  """
  Intercept term
  """
  intercept::Float64

  """
  Variables whose effects are random
  """
  random_effect_variables::Array{Symbol}

  """
  Variance components
  """
  variance_components::Vector{Float64}

  """
  Cross covariances for each of the covariance matrix
  """
  cross_covariances::Array{Matrix{Float64}}
end

"""
Simulate traits based on model specified in 'model' using data
stored in 'data_frame'.
"""
function simulate(model::Model, data_frame::DataFrame)
  # TODO: implement this function
end
\end{lstlisting}

\end{spacing}
%-----------------------------------------------------------------------------


\end{document}
